\documentclass[11pt, a4paper]{article}

\usepackage{amsmath}
\usepackage{amsfonts} %Matheschriften
\usepackage{amssymb} %Mathesymbole
%\usepackage{mathptmx} % Einstellung für Schriften und Sonderzeichen in mathematischen Umgebungen
                        % ändert SChriftfont
\usepackage{wasysym} % Stellt diverse Sonderzeichen bereit
\usepackage{siunitx}
\usepackage{float}
\usepackage{microtype}
\usepackage{graphicx}
\usepackage{hyperref}
\usepackage{xcolor}
\usepackage[section]{placeins}
% allows for temporary adjustment of side margins
\usepackage{changepage}
\usepackage{rotating}


\usepackage[ngerman]{babel}
\addto\captionsngerman{%
 \renewcommand{\abstractname}{Einleitung}}

\title{Versuch 3: Vakuum}
\author{Team 2-13: Jascha Fricker, Benedict Brouwer}

\begin{document}
    \maketitle

    \tableofcontents

    \newpage

    \section{Einleitung}
    In diesem Versuch wurden die verschiedenen Eigenschaften einer Vakuumpumpe untersucht. Dazu musste aber zuerst das Druckmessgerät kallibriert werden.

    \section{Theorie}

    \paragraph{Piranimeter}
    Die Wärmeleitfähigkeit eines Gases ist bei kleinen Drücken abhängig vom dessen Druck. Damit einhergehend ist die benötigte Leistung um einen sich im Messaufbau befindlichen Wolframdraht auf Temperatur zu halten Druckabhängig. Diese Abhängigkeit kann genuzt werden indem Wiederstand und Stromstärke gemessen werden um Rückschlüsse auf den Druck zu ziehen.

    \paragraph{Saugvermögen} Bei konstantem Druck kann das Saugvermögen $S$ durch die (negative) Volumenänderung $\Delta V_L$
    \begin{align}
        \underbrace{\frac{d(p_L V_L)}{dt}}_{konst} &= Q_S = Q_V = p_V \cdot S \\
        \Rightarrow S &= \left\lvert\frac{p_L \cdot \Delta V}{p_V \cdot \Delta t}\right\rvert
    \end{align}
    bestimmt werden. Dabei ist $Q_s$ die Saugleisung bei Luftdruck $p_L$ und $Q_V$ die Saugleisung an der Vakuumpumpe mit Druck $p_V$.

    \paragraph{effektives Saugvermögen}
    





    \section{Ergebnisse}

    \section{Diskussion}

    \bibliographystyle{plain}
    \bibliography{literature}

\end{document}