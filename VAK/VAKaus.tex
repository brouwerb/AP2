\documentclass[11pt, a4paper]{article}

\usepackage{amsmath}
\usepackage{amsfonts} %Matheschriften
\usepackage{amssymb} %Mathesymbole
%\usepackage{mathptmx} % Einstellung für Schriften und Sonderzeichen in mathematischen Umgebungen
                        % ändert SChriftfont
\usepackage{wasysym} % Stellt diverse Sonderzeichen bereit
\usepackage{siunitx}
\usepackage{float}
\usepackage{microtype}
\usepackage{graphicx}
\usepackage{hyperref}
\usepackage{xcolor}
\usepackage[section]{placeins}
% allows for temporary adjustment of side margins
\usepackage{changepage}
\usepackage{rotating}


\usepackage[ngerman]{babel}
\addto\captionsngerman{%
 \renewcommand{\abstractname}{Einleitung}}

\title{Versuch X: }
\author{Team 2-13: Jascha Fricker, Benedict Brouwer}

begin{document}
    \maketitle

    \tableofcontents

    \newpage

    \section{Einleitung}
    In diesem Versuch wurden die verschiedenen Eigenschaften einer Vakuumpumpe untersucht. Dazu musste aber zuerst das Druckmessgerät kallibriert werden.

    \section{Theorie}

    \paragraph{Piranimeter}
    Da die Wärmeleitung eines Gases abhängig vom Druck dessen ist, und der Strom durch den Wolframdraht abhängig ist von der benötigten Leisung um diesen auf Temperatur zu halten, kann durch die Messung des Stroms für einen konstanten Widerstand Rückschlüsse auf den Druck gemacht werden.

    \secton{Ergebnisse}

    \section{Diskussion}

    \bibliographystyle{plain}
    \bibliography{literature}

\end{document}