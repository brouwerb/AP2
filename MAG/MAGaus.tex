\documentclass[11pt, a4paper]{article}

\usepackage{amsmath}
\usepackage{amsfonts} %Matheschriften
\usepackage{amssymb} %Mathesymbole
%\usepackage{mathptmx} % Einstellung für Schriften und Sonderzeichen in mathematischen Umgebungen
                        % ändert SChriftfont
\usepackage{wasysym} % Stellt diverse Sonderzeichen bereit
\usepackage{siunitx}
\usepackage{float}
\usepackage{microtype}
\usepackage{graphicx}
\usepackage{hyperref}
\usepackage{xcolor}
\usepackage[section]{placeins}
% allows for temporary adjustment of side margins
\usepackage{changepage}
\usepackage{rotating}


\usepackage[ngerman]{babel}
\addto\captionsngerman{%
 \renewcommand{\abstractname}{Einleitung}}

\title{Versuch 4: Magnetismus}
\author{Team 2-13: Jascha Fricker, Benedict Brouwer}

\begin{document}
    \maketitle

    \tableofcontents

    \newpage

    \section{Einleitung}

    In diesem Versuch werden die Eigenschaften des Magnetfelds einer Spule mittels einer Hall-Sonde untersucht. Dabei wird der Einfluss verschiedener Ströme und eines Eisenkerns gemessen.

    \section{Theorie}

    Nach dem Biot-Savart-Gesetz kann das Magnetfeld $(x)$ auf der Symmetrieachse einer dünnen Ringspule mit Radius $R$ durch die Formel
    \begin{align}
        B(x) = \frac{\mu_0 \mu_r N}{2} \cdot \frac{R^2 I}{\left(x^2 + R^2\right)^{\frac{3}{2}}} \label{eq:BiotSavart}
    \end{align}
    beschreiben werden. Dabei durchfließt die Spule eine Stromstärke $I$ mit einer Windungszahl $N$. Das Material in der Spule hat eine Permeabilität $\mu_r$ (bei Luft $\mu_r = 1$).
    
    Durch Umstellung der Gleichung nach $x$ können bei gegebenem $B_max$ die Spulenränder
    \begin{align}
        x_{min, \ max} = \pm \sqrt{\left(\frac{\mu_0 \mu_r N}{2} \cdot \frac{R^2 I}{B_{max}}\right)^\frac{2}{3} - R^2} \label{eq:bmax}
    \end{align}
    bestimmt werden. 

    \section{Ergebnisse}
    \subsection{longitudinale Konfiguration}
    Die rohen Messwerte der 4x3 verschiedenen Messreihen der longitudinalen Konfiguration wurden im Graph \ref{fig:longmess} geplottet.
    \begin{figure}[h]
        \centering
        \includegraphics[width=0.9\textwidth]{raw1.pdf}
        \caption{Messwerte der longitudinalen Konfiguration}
        \label{fig:longmess}
    \end{figure}

    Im Plot ? wurde die Magnetfeldstärke abzüglich der Hintergrundmagnetisierung aufgetragen und gespiegelt. Anschließend wurde die Funktion \ref{eq:BiotSavart} auf die Messwerte außerhalb der Spule gefittet und auch aufgetragen. Mit einem abgelesenen $B_{max}$ und den gefitteten parametern wurden die Spulenränder berechnet. Diese werden als vertikale Balken in dem Graphen dargestellt. Die Ergebnisse des Fits und die errechneten Werte sind in Tabelle \ref{tab:fit} aufgelistet.
    \begin{table}[h]
        \centering
        \begin{tabular}{c | c | c}
            \textbf{Parameter} & \textbf{Wert mit 1A} & \textbf{Wert mit 1,5A} \\
            \hline
            $B_{max}$ & $8,26 \si{\milli\tesla}$ & $12,3 \si{\milli\tesla}$ \\
            $R_{eff}$ & $35,6 \si{\milli\metre}$ & $35,8 \si{\milli\metre}$ \\
            $I_{eff}$ & $0,74 \si{\ampere}$ & $1,10 \si{\ampere}$ \\
            $x_{min}$ & $25,9 \si{\milli\meter}$ & $25,8 \si{\milli\meter}$ \\
            $x_{max}$ & $-25,9 \si{\milli\meter}$ & $-25,9 \si{\milli\meter}$ \\
        \end{tabular}
        \caption{Ergebnisse Aufgabe 5.2.2}
        \label{tab:fit}
    \end{table}
    \begin{figure}
        \centering
        \includegraphics[width=0.8\textwidth]{fit1.pdf}
        \caption{Fit der longitudinalen Konfiguration}
        \label{fig:longfit}
    \end{figure}

    \subsection{Bestimmung von $\mu_r$}
    Mit den in \ref{tab:fit} aufgelisteten Ergebnissen, kann die Funktion \ref{eq:BiotSavart} mit einem freien Parameter $\mu_r$ auf die Messwerte der longitudinalen Konfiguration gefittet werden. So kann $\mu_r$ bestimmt werden.


    \subsection{transversale Konfiguration}
    Die gemessenen Daten der transversalen Konfiguration wurden im Graph \ref{fig:transmess} geplottet.
    Genauso wie bei der longitudinalen Konfiguration fällt das Magnetfeld mit größerem Abstand ab, beide fallen sogar ungefähr gleich schnell ab, nur ist erstaunlicherweise der Startwert der transversalen Messung mit $10 \si{\milli\tesla}$ direkt an der Seite der Spule größer als der Wert der longitudinalen Messung direkt in der Spule, der etwa $8 \si{\milli\tesla}$ beträgt.
    \begin{figure}[h]
        \centering
        \includegraphics[width=0.8\textwidth]{raw2.pdf}
        \caption{Messwerte transversale Konfiguration}
        \label{fig:transmess}
    \end{figure}


    \section{Diskussion}

    \bibliographystyle{plain}
    \bibliography{literature}

\end{document}