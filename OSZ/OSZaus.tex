\documentclass[11pt, a4paper]{article}

\usepackage{amsmath}
\usepackage{amsfonts} %Matheschriften
\usepackage{amssymb} %Mathesymbole
%\usepackage{mathptmx} % Einstellung für Schriften und Sonderzeichen in mathematischen Umgebungen
                        % ändert SChriftfont
\usepackage{wasysym} % Stellt diverse Sonderzeichen bereit
\usepackage{siunitx}
\usepackage{float}
\usepackage{microtype}
\usepackage{graphicx}
\usepackage{hyperref}
\usepackage{xcolor}
\usepackage[section]{placeins}
% allows for temporary adjustment of side margins
\usepackage{changepage}
\usepackage{rotating}


\usepackage[ngerman]{babel}
\addto\captionsngerman{%
 \renewcommand{\abstractname}{Einleitung}}

\title{Versuch 5: Oszilloskop}
\author{Team 2-13: Jascha Fricker, Benedict Brouwer}

\begin{document}
    \maketitle

    \tableofcontents

    \newpage

    \section{Einleitung}
    In diesem Versuch wurden verschiedene RC und LC-Schaltungen mit einem Oszilloskop gemessen.

    \section{Theorie}

    \subsection{Tieftpass}
    Eine Tiefpass besteht aus einem Kondensator mit Kapazität $C$ und einem Widerstand mit Wert $R$. Die Ausgangsspannung wird beim Tiefpass am Kondensator (beim Hochpass am Widerstand) abgegriffen.
    Beim Tiefpass lässt sich aus $R$, $C$ und der Frequenz $f$ die Durchgangskurve, also der Quoteint der Ausgangsspannung $U_{\text{Atp}}$ und der Eingangsspannung $U_{\text{E}}$ 
    \begin{align}
        g_{\text{tp}} &= \frac{U_{\text{Atp}}}{U_{\text{E}}} = \frac{1}{\sqrt{1 + \left( \omega R C \right)}} \\
    \end{align}
    berechnen. Auch die Phasenverschiebung
    \begin{align}
        \phi_{\text{tp}} &= \arctan{\frac{1}{\omega R C}} \\
    \end{align}
    lässt sich theoretisch berechnen.
    Die Grenzfrequenz, bei der die Phasenverschiebung 0 ist, ist somit
    \begin{equation}
        f_{\text{g}} = \frac{1}{2 \pi R C} \,.
    \end{equation}

    \subsection{Schwingkreis}
    Auch bei einem Serienschwingkreis (Serienschaltuung von Spule und Kondensator) kann die Durchgangskurve
    \begin{equation}
        g_{\text{Ssc}} = \frac{U_{\text{Atp}}}{U_{\text{E}}} = \frac{R}{\sqrt{R^2 + \left( \omega R - \frac{1}{\omega C} \right)^2}} \\
    \end{equation}
    mit dem Ausgangswiderstand $R$, die Phasenverschiebung
    \begin{equation}
        \tan{\phi_{\text{Ssc}}} = \frac{1}{R}\left( \omega L - \frac{1}{\omega C} \right) \\
    \end{equation}
    und die Eigenfreqenz
    \begin{equation}
        f_{\text{Ssc}} = \frac{1}{2 \pi} \sqrt{\frac{1}{L C} - \left(\frac{R}{2L}\right)^2} \,.
    \end{equation}
    berechnen.

    Bei einem Parallelschwinkreis lässt sich die Eingenschwingung
    \begin{equation}
        f_{\text{Psc}} \approx \frac{1}{2 \pi} \sqrt{\frac{1}{R L} - \left(\frac{R^4 C}{2L^3}\right)} \,.
    \end{equation}
    nur ungefähr berechnen.
    


    \section{Ergebnisse}

    \section{Diskussion}

    \bibliographystyle{plain}
    \bibliography{literature}

\end{document}