\documentclass[11pt, a4paper]{article}

\usepackage{amsmath}
\usepackage{amsfonts} %Matheschriften
\usepackage{amssymb} %Mathesymbole
%\usepackage{mathptmx} % Einstellung für Schriften und Sonderzeichen in mathematischen Umgebungen
                        % ändert SChriftfont
\usepackage{wasysym} % Stellt diverse Sonderzeichen bereit
\usepackage{siunitx}
\usepackage{float}
\usepackage{microtype}
\usepackage{graphicx}
\usepackage{hyperref}
\usepackage{xcolor}
\usepackage[section]{placeins}
% allows for temporary adjustment of side margins
\usepackage{changepage}
\usepackage{rotating}
\usepackage{tikz}
\usepackage{circuitikz}


\usepackage[ngerman]{babel}
\addto\captionsngerman{%
 \renewcommand{\abstractname}{Einleitung}}

\title{Versuch 2: Brückenschaltung}
\author{Team 2-13: Jascha Fricker, Benedict Brouwer}

\begin{document}
    \maketitle

    \tableofcontents

    \newpage

    \section{Einleitung}
    Durch die Brückenschaltung können Widerstände und Impedanzen sehr genau bestimmt werden. In diesem Versuch werden mit dieser Methode verschiedene, Widerstände, Spulen und Kondensatoren untersucht.
    \section{Experimenteller Aufbau}
    \begin{circuitikz}
        \ctikzset{european resistors}
        \draw (0,0)
        to[V,v=$U_q$] (0,4) % The voltage source
        to[short] (2,4)
        to[R=$Z_1$] (2,2) % The resistor
        to[R=$Z_2$] (2,0) % The resistor
        to[short] (0,0);
        \draw (2,0)
        to[short] (6,0)
        to[pR=$1k\si{\ohm}$] (6,4)
        to[short] (2,4);
        \draw (2,2)
        to[voltmeter] (5.5,2);
     \end{circuitikz}
    \section{Theorie}
    Durch das Potentiometer kann für jeden Widerstand 
    \input{wiepo.txt}
    \section{Ergebnisse}
    \paragraph{Aufgabe 7}


    \section{Diskussion}

    \bibliographystyle{plain}
    \bibliography{literature}

\end{document}