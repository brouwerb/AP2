\documentclass[11pt, a4paper]{article}

\usepackage{amsmath}
\usepackage{amsfonts} %Matheschriften
\usepackage{amssymb} %Mathesymbole
%\usepackage{mathptmx} % Einstellung für Schriften und Sonderzeichen in mathematischen Umgebungen
                        % ändert SChriftfont
\usepackage{wasysym} % Stellt diverse Sonderzeichen bereit
\usepackage{siunitx}
\usepackage{float}
\usepackage{microtype}
\usepackage{graphicx}
\usepackage{hyperref}
\usepackage{xcolor}
\usepackage[section]{placeins}
% allows for temporary adjustment of side margins
\usepackage{changepage}
\usepackage{rotating}
\usepackage{tikz}
\usepackage{circuitikz}


\usepackage[ngerman]{babel}
\addto\captionsngerman{%
 \renewcommand{\abstractname}{Einleitung}}

\title{Versuch 2: Brückenschaltung}
\author{Team 2-13: Jascha Fricker, Benedict Brouwer}

\begin{document}
    \maketitle

    \tableofcontents

    \newpage

    \section{Einleitung}
    Durch die Brückenschaltung können Widerstände und Impedanzen sehr genau bestimmt werden. In diesem Versuch werden mit dieser Methode verschiedene, Widerstände, Spulen und Kondensatoren untersucht.
    \section{Experimenteller Aufbau}
    \begin{circuitikz}
        \ctikzset{european resistors}
        \draw (0,0)
        to[V,v=$U_q$] (0,4) % The voltage source
        to[short] (2,4)
        to[R=$Z_1$] (2,2) % The resistor
        to[R=$Z_2$] (2,0) % The resistor
        to[short] (0,0);
        \draw (2,0)
        to[short] (6,0)
        to[pR=$1k\si{\ohm}$] (6,4)
        to[short] (2,4);
        \draw (2,2)
        to[ammeter] (5.5,2);
        \label{circuit}
     \end{circuitikz}
    \section{Theorie}
    In diesem Versuch werden mithilfe der in \ref{circuit} gezeigten Brückenschaltung verschiedene Widerstände, Spulen und Kondensatoren untersucht.
    Wenn kein Strom durch das Ampèremeter fließt, bzw der Graph auf dem Oszilloskop horizontal ist, gilt die Schaltung als abgeglichen und es gilt das Verhähltnis
    \begin{align}
        \frac{Z_1}{Z_2} = \frac{R_3}{R_4} = \frac{R_p}{1\si{\kilo\ohm} - R_p} \\
        \Rightarrow Z_1 = \frac{R_p}{1\si{\kilo\ohm} - R_p} \cdot Z_2 \,,
    \end{align}
    wobei $R_p$ der Ablesewert des Potentiometers ist, und $Z_2$ der bekannte (komplexe) Vergleichwiderstand.

    Wenn eine Spule gemessen wird, gilt speziell
    \begin{align}
        R_1 &= \frac{A}{1\si{\kilo\ohm} - A} \cdot \left(R_S + R_V\right) \\
        L_1 &= \frac{A}{1\si{\kilo\ohm} - A} \cdot \left(L_S\right) \\
        \text{mit} \ \ Z_2 &= R_2 + j\omega L_2 = R_V + R_S + j\omega L_S
    \end{align}
    
    \section{Ergebnisse}
    \paragraph{Aufgabe 7}
    Durch das Potentiometer kann für jeden Widerstand 
    \input{wiepo.txt}

    \paragraph{Aufgabe 8}

    \input{wiespu.txt}

    \section{Diskussion}

    \section{Anhang}
    \subsection{Messwerte Aufgabe 9 und 10}
    \input{eigglueh.txt}

    \bibliographystyle{plain}
    \bibliography{literature}

\end{document}