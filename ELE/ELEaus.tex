\documentclass[11pt, a4paper]{article}

\usepackage{amsmath}
\usepackage{amsfonts} %Matheschriften
\usepackage{amssymb} %Mathesymbole
%\usepackage{mathptmx} % Einstellung für Schriften und Sonderzeichen in mathematischen Umgebungen
                        % ändert SChriftfont
\usepackage{wasysym} % Stellt diverse Sonderzeichen bereit
\usepackage{siunitx}
\usepackage{float}
\usepackage{microtype}
\usepackage{graphicx}
\usepackage{hyperref}
\usepackage{xcolor}
\usepackage[section]{placeins}
% allows for temporary adjustment of side margins
\usepackage{changepage}
\usepackage{rotating}


\usepackage[ngerman]{babel}
\addto\captionsngerman{%
 \renewcommand{\abstractname}{Einleitung}}

\title{Versuch 1: Eigenschaften des Elektron}
\author{Team 2-13: Jascha Fricker, Benedict Brouwer}
\begin{document}
    \maketitle


    \tableofcontents

    \newpage

    \section{Einleitung}

    Bei diesem Versuch werden die Eigenschaften des Elektrons betrachtet und dazu zwei Experimente durchgeführt. Zunächst wird duch das Fadenstrahlrohr
    der Quotient $\frac{e}{m}$ (spezifische Elektronenladung) berechnet. Durch den Milikam-Versuch kann anschließend die Elementarladung $e$ bestimmt
    werden, sodass auch die Masse des Elektrons $m_e$ berechnet werden kann.

    \section{Bestimmung der spezifischen Elektronenladung}

    \subsection{Theorie}

    Im Fadenstrahlrohr werden die Elektronen durch ein elektrisches Feld beschleunigt. Die Endgeschwindigkeit kann durch gleichsetzten der Energien bestimmt werden.

    \begin{align}
    \frac{mv^2}{2} = E_{kin} &= E_{elek} = q \cdot U \label{geschw}\\
    \Rightarrow v &= \sqrt{\frac{2qU}{m}}
    \end{align}
    
    Das Magnetfeld $B$ der Helmholzspulen kann mithilfe der Biot-Savart-Gesetzes bestimmt werden.
    Mit dem Strom $I$, der Windungszahl $N$ und dem Radius $R$ ergibt sich für diesen Versuch:

    \begin{align}
        B = \frac{\mu_0 N I}{R} \cdot \frac{4}{5}^{\frac{3}{2}}
    \end{align}

    Die spezifische Elektronenladung ist der Quotient aus Ladung und Masse $\frac{e}{m}$.
    Diese kann durch die Messung des Radius des Strahls im Fadenstrahlrohr bestimmt werden. Es gilt:

    \begin{align}{}
        \frac{mv^2}{r} = F_{rot} &= F_{mag} = q \cdot v \cdot B \\
        \overset{\text{(\ref{geschw})}}{\Rightarrow} \ \  \frac{q}{m} &= \frac{2U}{B^2 \cdot r^2} \\
    \end{align}

    \subsection{Ergebnisse}
    \paragraph{Vorüberlegungen}

    Aus einer Beschleunigungsspannung von maximal $300 \si{V}$ kann die maximale Geschwindigkeit eines Elektrons mit Ladung $e$ im nichtrelativistischen Fall
    \begin{align}
        v = \sqrt{\frac{2 e U}{m_e}} = 1,02 \cdot 10^{7} \si{m/s} < 2,9 \cdot 10^{7} \si{m/s} = 10\% \cdot c
    \end{align}

    berechnet werden, da diese Kleiner als zehn Prozent der Lichtgeschwindigkeit ist, kann im weiteren auch nichtrelativistisch gerechnet werden.

    Jetzt muss noch überprüft werden, ob die thermische Energie der Glühkathode die Messungen verfälschen könnte.
    \begin{align}
        v_{tmax} = \frac{v_{100V}}{100} = \frac{\sqrt{\frac{2 e \cdot 100 \si{V}}{m_e}}}{100} = 59310 \si{m \per s} \\
        E_{tmax} = \frac{m_e \cdot v_{tmax}^2}{2} = 1,602 \cdot 10^{-21} \si{J} = \frac{3}{2} k T_{max} \\
        \Rightarrow T_{max} = \frac{2 E_{tmax}}{3 k} = 77K
    \end{align}
    Die thermische Energie plus die Austrittsarbeit muss kleiner als $E_{tmax}$ sein, da sonst die Messungen verfälscht würden. Da die Austrittsarbeit des Material leider nicht bekannt ist, kann die eigenlich maximale Temperatur aber nicht bestimmt werden.

    \paragraph{Magnetfeld}

    

    \section{Diskussion}

    \bibliographystyle{plain}
    \bibliography{literature}

\end{document}