\documentclass[11pt, a4paper]{article}

\usepackage{amsmath}
\usepackage{amsfonts} %Matheschriften
\usepackage{amssymb} %Mathesymbole
%\usepackage{mathptmx} % Einstellung für Schriften und Sonderzeichen in mathematischen Umgebungen
                        % ändert SChriftfont
\usepackage{wasysym} % Stellt diverse Sonderzeichen bereit
\usepackage{siunitx}
\usepackage{float}
\usepackage{microtype}
\usepackage{graphicx}
\usepackage{hyperref}
\usepackage{xcolor}
\usepackage[section]{placeins}
% allows for temporary adjustment of side margins
\usepackage{changepage}
\usepackage{rotating}


\usepackage[ngerman]{babel}
\addto\captionsngerman{%
 \renewcommand{\abstractname}{Einleitung}}

\title{Versuch 1: Eigenschaften des Elektron}
\author{Team 2-13: Jascha Fricker, Benedict Brouwer}
\begin{document}
    \maketitle


    \tableofcontents

    \newpage

    \section{Einleitung}

    Bei diesem Versuch werden Elektronenladung bzw -Masse und Elementarladung bestimmt. Ersteres durch die Ablenkung eines Elektronenstrahls im
    Fadenstrahlrohr, letzteres durch den Millikan-Versuch.

    \section{Bestimmung der spezigischen Elektronenladung}

    \subsection{Theorie}

    Im Fadenstrahlrohr werden die Elektronen durch ein elektrisches Feld beschleunigt. Die Endgeschwindigkeit kann durch gleichsetzten der Energien bestimmt werden.

    \begin{align}
    \frac{mv^2}{2} = E_{kin} = E_{elek} = q \cdot U \label{geschw}\\
    \end{align}
    Die spezifische Elektronenladung ist der Quotient aus Ladung und Masse $\frac{e}{m}$.
    Diese kann durch die Messung des Radius des Strahls im Fadenstrahlrohr bestimmt werden. Es gilt:

    \begin{align}{}
        \frac{mv^2}{r} = F_{rot} &= F_{mag} = q \cdot v \cdot B \\
        \overset{\text{(\ref{geschw})}}{\Rightarrow} \ \  \frac{q}{m} &= \frac{2U}{B^2 \cdot r^2} \\
    \end{align}


    \section{Ergebnisse}

    \section{Diskussion}

    \bibliographystyle{plain}
    \bibliography{literature}

\end{document}